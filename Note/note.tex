\documentclass[12pt,A4paper]{article}
%\pdfoutput=1

\usepackage{multirow}
\usepackage{cite}
\usepackage{hyperref}
\usepackage{slashed}
\usepackage{graphicx}
    \usepackage{amsmath}
    \usepackage{caption}
    \usepackage{subcaption}
    \usepackage{cancel}
    \usepackage{etoolbox} % provides \patchcmd macro
    \makeatletter % modify the "headings" page style
    \patchcmd{\ps@headings}{{\slshape
ightmark}\hfil	hepage}{	hepage\hfil}{}{}
    \makeatother
    \pagestyle{headings}
\usepackage{listings}
\usepackage{color}
\usepackage{appendix}

\textwidth 170mm
\textheight 220mm
\oddsidemargin -5mm
\evensidemargin 5mm
\topmargin -16pt

\newcommand{\colspacea}{\hphantom{33333333}}
\newcommand{\colspaceb}{\hphantom{33333333}}
\newcommand{\colspacec}{\hphantom{33333333}}
\newcommand{\colspace}{\hphantom{33333333333}}


\newcommand{\xspace}{~}
\newcommand{\GeV}{\ensuremath{\,\text{Ge\hspace{-.08em}V}}\xspace}
\newcommand{\met}{\ensuremath{E_{T}^\mathrm{miss}}\xspace}
\newcommand{\PSGczDo}{\ensuremath{\widetilde{\chi}^{0}_{1}}\xspace} % neutralino
\newcommand{\Pp}{\ensuremath{p}}
\newcommand{\PSg}{\ensuremath{\widetilde{g}}}
\newcommand{\PQq}{\ensuremath{q}}
\newcommand{\cPqt}{\ensuremath{t}}
\newcommand{\cPq}{\ensuremath{q}}
\newcommand{\cPqb}{\ensuremath{b}}
\newcommand{\PAQq}{\overline{\ensuremath{q}}}
\newcommand{\cPaqt}{\overline{\ensuremath{t}}}
\newcommand{\cPaqb}{\overline{\ensuremath{b}}}
\newcommand{\PSQ}{\ensuremath{\widetilde{q}}}
\newcommand{\PSGcz}{\ensuremath{\widetilde{\chi_0^1}}}
\newcommand{\qqbar}{\ensuremath{\PQq\PAQq}\xspace}
%\newcommand{\mgluino}{\ensuremath{m_{\PSg}}\xspace}
\newcommand{\HT}{\ensuremath{H_{\mathrm T}}\xspace}
\newcommand{\MHT}{\ensuremath{H_{\mathrm T}^{\text{miss}}}\xspace}
\newcommand{\njets}{\ensuremath{N_{\text{jet}}}~}
\newcommand{\nbjets}{\ensuremath{N_{\text{b-jet}}}~}
\newcommand{\mgluino}{\ensuremath{m_{\PSg}}\xspace}
\newcommand{\msquark}{\ensuremath{m_{\PSQ}}\xspace}
%\newcommand{\mlsp}{\ensuremath{m_{{\PSGcz}_1}}\xspace}
\newcommand{\tbbar}{\ensuremath{\cPqt\cPaqb}\xspace} % t-bbar
\newcommand{\ttbar}{\ensuremath{t\overline{t}}\xspace} % t-tbar
\newcommand{\bbbar}{\ensuremath{b\overline{b}}\xspace} % b-bbar
\newcommand{\sTop}{\ensuremath{\widetilde{t}}\xspace}
\newcommand{\sBot}{\ensuremath{\widetilde{b}}\xspace}
\newcommand{\sQua}{\ensuremath{\widetilde{q}}\xspace}
\newcommand{\PASQt}{\ensuremath{\overline{\widetilde{t}}}\xspace} % anti stop
\newcommand{\PASQb}{\ensuremath{\overline{\widetilde{b}}}\xspace} % anti sbottom
\newcommand{\PASQ}{\ensuremath{\overline{\widetilde{q}}}\xspace} % anti squark




\newcommand{\neles}{\ensuremath{N_{\text{electron}}}\xspace}
\newcommand{\nmuons}{\ensuremath{N_{\text{muon}}}\xspace}
\newcommand{\nisomuons}{\ensuremath{N_{\text{isolated tracks}}^{\text{(muon)}}}\xspace}
\newcommand{\nisoeles}{\ensuremath{N_{\text{isolated tracks}}^{\text{(electron)}}}\xspace}
\newcommand{\nisohads}{\ensuremath{N_{\text{isolated tracks}}^{\text{(hadron)}}}\xspace}
\newcommand{\dpmht}[1]{\ensuremath{\Delta\phi_{\MHT,j_{#1}}}\xspace}


\newcommand{\mulobs}{$\sigma^{\mathrm{UL}}_{\mathrm{obs}}$}
\newcommand{\ulobs}{\sigma^{\mathrm{UL}}_{\mathrm{obs}}}
\newcommand{\mulexp}{$\sigma^{\mathrm{UL}}_{\mathrm{exp}}$}
\newcommand{\ulexp}{\sigma^{\mathrm{UL}}_{\mathrm{exp}}}
\newcommand{\mtheo}{$\sigma^{\mathrm{pred}}_{\mathrm{theo}}$}
\newcommand{\theo}{\sigma_{\mathrm{pred}}^{\mathrm{theo}}}

\newcommand{\lumi}{\mathcal{L}}
\newcommand{\twomm}{\hspace{2mm}}
\newcommand{\neu}{{\widetilde{\chi}_1^0}}

\newcommand{\go}{{\widetilde{g}}}
\newcommand{\sq}{{\widetilde{q}}}
\newcommand{\mmsq}{$m_{\widetilde{q}}$}
\newcommand{\msq}{m_{\widetilde{q}}}
\newcommand{\mlsp}{m_{\widetilde{\chi}}}
\newcommand{\mmlsp}{$m_{\widetilde{\chi}}$}
\newcommand{\mgo}{m_{\widetilde{g}}}
\newcommand{\mmgo}{$m_{\widetilde{g}}$}

\definecolor{dkgreen}{rgb}{0,0.6,0}
\definecolor{gray}{rgb}{0.5,0.5,0.5}
\definecolor{mauve}{rgb}{0.58,0,0.82}
\lstset{ %
  language=C++,                % the language of the code
  basicstyle=\footnotesize,           % the size of the fonts that are used for the code
  %  numbers=left,                   % where to put the line-numbers
  %  numberstyle=\tiny\color{gray},  % the style that is used for the line-numbers
  %  stepnumber=2,                   % the step between two line-numbers. If it's 1, each line
  % will be numbered
  %  numbersep=5pt,                  % how far the line-numbers are from the code
  backgroundcolor=\color{white},      % choose the background color. You must add \usepackage{color}
  showspaces=false,               % show spaces adding particular underscores
  showstringspaces=false,         % underline spaces within strings
  showtabs=false,                 % show tabs within strings adding particular underscores
  frame=single,                   % adds a frame around the code
  rulecolor=\color{black},        % if not set, the frame-color may be changed on line-breaks within not-black text (e.g. commens (green here))
  tabsize=2,                      % sets default tabsize to 2 spaces
  captionpos=b,                   % sets the caption-position to bottom
  breaklines=true,                % sets automatic line breaking
  breakatwhitespace=false,        % sets if automatic breaks should only happen at whitespace
  title=\lstname,                   % show the filename of files included with \lstinputlisting;
  % also try caption instead of title
  keywordstyle=\color{blue},          % keyword style
  commentstyle=\color{dkgreen},       % comment style
  stringstyle=\color{mauve},         % string literal style
  escapeinside={\%*}{*)},            % if you want to add a comment within your code
  morekeywords={*,\dots}               % if you want to add more keywords to the set
}


\title{Validation of the \texttt{MadAnalysis 5} implementation of CMS-EXO-16-052}
\author{Dajeong Jeon, Seulgi Kim, Daniel Lee, Ian J. Watson, Sam Bein, Jory Sonneveld \\
\normalsize {\it email: ian.james.watson@cern.ch}
\date{\today}
}

\begin{document}
        \maketitle

%\clearpage
%%%%%%%%%%%%%%%%%%%%%%%%%%%%%%%%%%%%%%%%%%%%%%%%%
%%%%%%%%%%%%%%%    Setup   %%%%%%%%%%%%%%%%%%%%%%
%%%%%%%%%%%%%%%%%%%%%%%%%%%%%%%%%%%%%%%%%%%%%%%%%

%\section{Setup}
%In this document, the \texttt{MadAnalysis 5} implementation of the
%dilepton plus \met\ analysis of CMS at the LHC at
%$\sqrt{s}=13$ TeV
%\href{http://cms-results.web.cern.ch/cms-results/public-results/preliminary-results/EXO-16-052/}{CMS-EXO-16-052}
%(see also \href{http://arxiv.org/abs/XXXXX.XXXX}%{arXiv:XXXXX.XXXX}) is
%validated.

\section{INTRODUCTION}

\section{SIMULATION DETAILS}
\begin{table}[htb]
\centering
\vspace{5pt}
\begin{tabular}{|c||l|l|}
\hline
Selections & MG5 & Offical \\
\hline
cut1 & 0.9 & 0.9 \\
\hline
cut2 & 0.9 & 0.9 \\
\hline
cut3 & 0.9 & 0.9 \\
\hline
cut4 & 0.9 & 0.9 \\
\hline
\end{tabular}
\caption{Cut flows, expressed in terms of efficiencies, for three signal samples in signal region SR2jl}\label{tab:cutflow}

\end{table}
\section{RESULTS}
\subsection{Cut-flow}


\subsection{Various Samples}

Theory \(\sigma\) is the cross-section output from MadGraph.

% \(A \times e\)
The efficiency is determined from the fraction of events reconstructed by Delphes and passing the analysis cuts into the signal region.

\(N_{exp}\) is the number of events resulting from the expected cross-section and efficiency for the CMS analysis using 34 fb\(^{-1}\).

\begin{figure}
\begin{center}
\resizebox{\columnwidth}{!}{%
\begin{tabular}{lrrr}
Sample & \(\sigma\) Cross Section (pb) & Efficiency & \(N_{Expected}\)\\
\hline
./Samples/DMsimp\_s\_spin1-axial\_0j.txt & 0.021318904968 & 0.1272 & 97.3523531583\\
./Samples/DMsimp\_s\_spin1-axial\_ee\_0j.txt & 0.010659482475 & 0.0978 & 37.4256561594\\
./Samples/DMsimp\_s\_spin1-axial\_mm\_0j.txt & 0.010659482475 & 0.1481 & 56.6742298283\\
./Samples/DMsimp\_s\_spin1-vector\_0j.txt & 0.0393553945622 & 0.1227 & 173.357758169\\
./Samples/DMsimp\_s\_spin1-vector\_0j\_qcd.txt & 0.06788 & 0.1251 & 304.8551892\\
./Samples/DMsimp\_s\_spin1-vector\_1j.txt & 0.0666819164253 & 0.0752 & 180.037839919\\
./Samples/DMsimp\_s\_spin1-vector\_1j\_qcd.txt & 0.1251 & 0.1125 & 505.082719261\\
./Samples/DMsimp\_s\_spin1-vector\_ee\_0j.txt & 0.0196775973111 & 0.1003 & 70.8545020699\\
./Samples/DMsimp\_s\_spin1-vector\_mm\_0j.txt & 0.0196775973111 & 0.1572 & 111.050126873\\
\end{tabular}%
}
\caption{Samples of spin-1 axial and vector DM models. 0j is for
  events generated without extra partons generated by MAdGraph, 1j for
  is for 1 extra parton generated by MadGraph, and QCD is including
  QCD NLO calculations in the MadGraph calcualtion. ee is for Z to
  electrons only, mm for muons, otherwise both the electron and muon
  channels are generated.}
\end{center}
\end{figure}

The exclusion limit is determined by taking the 95\% C.L. upper limit
based on the number of expected events and the results of CMS

\begin{figure}
\begin{center}

\resizebox{\columnwidth}{!}{%
  \begin{tabular}{lrrrr}
Sample & Cross Section (pb) & Eff. & Expected & Exclusion (sigma/sigma\(_{\text{0}}\))\\
\hline
Run 30/10/2017 &  &  &  & \\
\hline
./Samples/DMsimp\_s\_spin0P\_0j\_1000\_1.txt & 5.28328517293\,(-06) & 0.0000 & 0.0 & 99999.9996275\\
./Samples/DMsimp\_s\_spin0P\_0j\_100\_1.txt & 5.29581021579\,(-05) & 0.0000 & 0.0 & 99999.9996275\\
./Samples/DMsimp\_s\_spin0P\_0j\_10\_1.txt & 9.81497810099\,(-05) & 0.0000 & 0.0 & 99999.9996275\\
./Samples/DMsimp\_s\_spin0P\_0j\_500\_1.txt & 2.10019490453\,(-05) & 0.0000 & 0.0 & 99999.9996275\\
./Samples/DMsimp\_s\_spin0P\_0j\_50\_1.txt & 8.02167613779\,(-05) & 0.0000 & 0.0 & 99999.9996275\\
./Samples/DMsimp\_s\_spin0S\_0j\_1000\_1.txt & 6.46834888936\,(-06) & 0.0000 & 0.0 & 99999.9996275\\
./Samples/DMsimp\_s\_spin0S\_0j\_100\_1.txt & 9.65746285084\,(-05) & 0.0000 & 0.0 & 99999.9996275\\
./Samples/DMsimp\_s\_spin0S\_0j\_10\_1.txt & 0.00028727155715 & 0.0000 & 0.0 & 99999.9996275\\
./Samples/DMsimp\_s\_spin0S\_0j\_500\_1.txt & 2.66324370678\,(-05) & 0.0000 & 0.0 & 99999.9996275\\
./Samples/DMsimp\_s\_spin0S\_0j\_50\_1.txt & 0.000177391817776 & 0.0000 & 0.0 & 99999.9996275\\
./Samples/DMsimp\_s\_spin0\_default.txt & 6.50211256127\,(-06) & 0.0000 & 0.0 & 99999.9996275\\
./Samples/DMsimp\_s\_spin1-axial\_0j.txt & 0.021318904968 & 0.1272 & 97.3523531583 & 0.864208980016\\
./Samples/DMsimp\_s\_spin1-axial\_ee\_0j.txt & 0.010659482475 & 0.0978 & 37.4256561594 & 2.24554648455\\
./Samples/DMsimp\_s\_spin1-axial\_mm\_0j.txt & 0.010659482475 & 0.1481 & 56.6742298283 & 1.48558733966\\
./Samples/DMsimp\_s\_spin1-vector\_0j.txt & 0.0393553945622 & 0.1227 & 173.357758169 & 0.484974465556\\
./Samples/DMsimp\_s\_spin1-vector\_0j\_10\_1.txt & 0.299915034487 & 0.0076 & 81.8288180094 & 1.02663162079\\
./Samples/DMsimp\_s\_spin1-vector\_0j\_50\_1.txt & 0.217641714014 & 0.0263 & 205.490777121 & 0.410468667038\\
./Samples/DMsimp\_s\_spin1-vector\_0j\_100\_1.txt & 0.168503454019 & 0.0490 & 296.414425965 & 0.283063751571\\
./Samples/DMsimp\_s\_spin1-vector\_0j\_500\_1.txt & 0.0419776079768 & 0.1291 & 194.553199914 & 0.430585232638\\
./Samples/DMsimp\_s\_spin1-vector\_0j\_1000\_1.txt & 0.0112635212815 & 0.1509 & 61.0179864735 & 1.37606381584\\
./Samples/DMsimp\_s\_spin1-vector\_0j\_qcd.txt & 0.06788 & 0.1251 & 304.8551892 & 0.275613171719\\
./Samples/DMsimp\_s\_spin1-vector\_1j.txt & 0.0666819164253 & 0.0752 & 180.037839919 & 0.469328247868\\
./Samples/DMsimp\_s\_spin1-vector\_1j\_qcd.txt & 0.1251 & 0.1125 & 505.082719261 & 0.166834705882\\
./Samples/DMsimp\_s\_spin1-vector\_1j\_qcd\_Qcut10.txt & 0.1251 & -1.0000 & -4491.09 & ERR\\
./Samples/DMsimp\_s\_spin1-vector\_1j\_qcd\_Qcut100.txt & 0.1251 & -1.0000 & -4491.09 & ERR\\
./Samples/DMsimp\_s\_spin1-vector\_ee\_0j.txt & 0.0196775973111 & 0.1003 & 70.8545020699 & 1.1868190876\\
./Samples/DMsimp\_s\_spin1-vector\_mm\_0j.txt & 0.0196775973111 & 0.1572 & 111.050126873 & 0.759155804105\\
\hline
Run 30/11/2017 &  &  &  & \\
\hline
./Samples/DMsimp\_s\_spin0P\_0j\_1000\_1.txt & 5.28486997514\,(-06) & 0.0064 & 0.00121425172549 & 69140.5077591\\
./Samples/DMsimp\_s\_spin0P\_0j\_100\_1.txt & 5.29739882874\,(-05) & 0.0181 & 0.0344219678493 & 2447.54969572\\
./Samples/DMsimp\_s\_spin0P\_0j\_100\_1\_QCD.txt & 0 & -1.0000 & -0.0 & 99999.9996275\\
./Samples/DMsimp\_s\_spin0P\_0j\_10\_1.txt & 9.8179223129\,(-05) & 0.0024 & 0.00845912186479 & 9942.6415468\\
./Samples/DMsimp\_s\_spin0P\_0j\_10\_1\_QCD.txt & 0.0001969 & -1.0000 & -7.06871 & ERR\\
./Samples/DMsimp\_s\_spin0P\_0j\_500\_1.txt & 2.10082494163\,(-05) & 0.0126 & 0.00950287154097 & 8841.97930621\\
./Samples/DMsimp\_s\_spin0P\_0j\_50\_1.txt & 8.0240824427\,(-05) & 0.0077 & 0.0221809710964 & 3788.30687204\\
./Samples/DMsimp\_s\_spin0P\_0j\_50\_1\_QCD.txt & 0.0001702 & -1.0000 & -6.11018 & ERR\\
./Samples/DMsimp\_s\_spin0S\_0j\_1000\_1.txt & 6.47028916275\,(-06) & 0.0000 & 0.0 & 99999.9996275\\
./Samples/DMsimp\_s\_spin0S\_0j\_100\_1.txt & 9.660359773\,(-05) & 0.0000 & 0.0 & 99999.9996275\\
./Samples/DMsimp\_s\_spin0S\_0j\_10\_1.txt & 0.0002873577256 & 0.0000 & 0.0 & 99999.9996275\\
./Samples/DMsimp\_s\_spin0S\_0j\_500\_1.txt & 2.66404256188\,(-05) & 0.0000 & 0.0 & 99999.9996275\\
./Samples/DMsimp\_s\_spin0S\_0j\_50\_1.txt & 0.000177445033 & 0.0000 & 0.0 & 99999.9996275\\
\end{tabular}%
}
  \caption{As before, with 0P and 0S being for pseudo-scalar and scalar DM models respectively.}
\end{center}
\end{figure}

\section{CONCLUSION}



\end{document}
